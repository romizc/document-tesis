\chapter{Related Works}
\label{chapter_Related_Works}

Spatio Temporal phenomena are present in almost every field of research, climate science, geographic information science (GIS), and others. With the advancement of technology it is possible to obtain exorbitant amounts of spatiotemporal data that represent these phenomena \cite{Atluri2018}. In this work we are interested in the representation of the phenomenon by means of univariate time series, temperature, energy consumption, econometrics.

Analyzing and studying space-time phenomena through data is a complex task and requires various statistical techniques and methods \cite{Rao2008}, in the particular case of univariate time series, it is not the exception depending on the type of analysis and information to be extracted. There are several tools, from those that were developed decades ago with a base of analysis and statistical methods, which are used with success until today, and with the advent of deep learning tools to process voluminous data, we now find more efficient techniques and models, but at a computational cost - in processing and computation - quite high.

%TODO Intro
When considering spatiotemporal data exists several methods to extract knowledge about the phenomena analyzed, in this work we consider the following tasks clustering, classification


\section{Spatiotemporal Modeling with Temporal Approach}
\label{Sec:SPT-Temporal}

Meteorological readings, hydrological parameters and many measures of air, soil and water pollution are often collected for a certain span, regularly in time, and at different survey stations of a monitoring network. Then, these observations can be viewed as realizations of a random function with a spatiotemporal variability. In this context, the arrangement of valid models for spatiotemporal prediction and environmental risk assessment is strongly required. Spatio-temporal models might be used for different goals: optimization of sampling design network, prediction at unsampled spatial locations or unsampled time points and computation of maps of predicted values, assessing the uncertainty of predicted values starting from the experimental measurements, trend detection in space and time, particularly important to cope with risks coming from concentrations of hazardous pollutants. Hence, more and more attention is given to spatiotemporal analysis in order to sort out these issues.

Applications of spatial statistics cover many areas and much of the original impetus for the area was driven by geostatistics but in recent years the applications have extended to sociology, economics, environmental and ecological sciences.

\section{Time-Series Clustering and Classification}
\label{Sec:ClusteringRelatedWorks}

% Time series clustering basics review 
During several decades the data mining community focused on time-series data, clustering is of particular interest in temporal data mining since it provides a mechanism to automatically find some structure in large data sets that would be otherwise difficult to summarize or visualize. In time were developed several clustering algorithms, criteria for evaluating the performance of results, and the measures to determine the similarity/dissimilarity between two time series being compared, either in the forms of raw data, extracted features, or some model parameters \cite{Liao2005, Aghabozorgi2015}.

We review works interested in methods concerning manipulating raw sequential data sets (univariate time series), using Partitioning Methods \cite{Kaufman2009}, in accordance with the purpose of grouping the data, it is necessary to consider a dis/similarity measure to compare the time series. Traditionally, the Euclidean distance were used to compare two time-series and measure the dis/similarity, but the nature of time--series ask for a more robust measure that is able to capture the temporal variability. Considering the dis/similarity focused on the shape between two time-series, the Dynamic Time Warping (DTW) measure, which originally was used for speech recognition, and later introduced to time-series for similarity measurement \cite{Sakoe1978}.

The use of DTW as a similarity measure for time series gained popularity through the years, due to its computational cost, there are several variations of the original version that reduce the computational cost. In 

% Success Cases using DTW as a similarity measure
Dynamic  Time  Warping(DTW) – was introduced in 1978 in the context of speech recognition \cite{Sakoe1978}. DTW is a way to define the distance measure between two univariate time series, and it can be used with a $k$-Nearest Neighbors ($kNN$) classifier. The main idea consists in calculating the distance between two time series not point wisely, like it is done using Euclidean distance, but using the mapping between structurally similar points, which is time-invariant.

% Time Series Classification
Techniques and methods for Time--Series classification in the last decades presented great advances due to the use of sequential models, in particular deep neural networks \cite{Fawaz2019}. 

Extensive  experimental  evaluation  [1] showed  that  1-NN  DTW  is  a  strong  benchmark  which  cannot  be  beaten  by many proposed algorithms. If beaten, the advantage is often not so significant,compared to the difficulty of the implementation or the computation.

In the same manner exists several approaches to process the time-series Geurts (2001) proposes to classify time series data based on combining local properties or patterns in the time series. Zhang et al. (2004) develop a representation method using wavelet decomposition that can automatically select the parameters for the classification task. 

They propose a nearest neighbor classification algorithm, using the derived appropriate scale. Kadous and Sammut (2005) use metafeature approach (i.e. recurring substructure) like local maxima in time series to generate classifiers. Similarly, Yang et al. (2005) focus on feature subset selection (FSS) based on common principal components, which is called CleVer, to retain the correlation information among original features. Classification is employed to evaluate the effectiveness of the selected subset of features. 
The use of neural networks for time-series classification

 While the complexity is far higher than its static, non-spatial counterpart the ideas behind spatio-temporal clustering are similar - that is, either characteristic features of objects in a spatio-temporal region or the spatio-temporal characteristics of a set of objects are sought (Ng and Han 1994; Ng 1996).

\section{Uni-Variate Time Series Analysis}
\label{Sec:TSAnalysis}
% Review the process of modelinng spatiotemporal data, focusing in simple methods that are very expensive 

% Modelos para problemas espacio-temporales:
In several domains, including climate science, neuroscience, social sciences, epidemiology, transportation, criminology, and Earth sciences, space and time are the main characteristic to consider in its observational data in the form of time--series. These aspects along with the technical advantages of nowadays, allow to domain specialists collect a vast amounts of data in order to analyze and study certain phenomenon with great detail \cite{}. 

Time series analysis has quite a long history. Techniques for statistical modeling and spectral analysis of real or complex-valued time series have been in use for more than fifty years \cite{Hyndman2006, Chatfield2019}. Weather forecasting, financial or stock market prediction and automatic process control have been some of the oldest and most studied applications of such time series analysis \cite{Box1976}. These applications saw the advent of an increased role for machine learning techniques like Hidden Markov Models and time-delay neural networks in time series analysis.

The problem of future traffic prediction can be regarded as time series modelling and forecasting. Common time series data include financial data and temperature data. According to the number of independent variables, time series analysis problems can be divided into univariate time series analysis and multivariate time series analysis. In univariate time series, the current value of the series is only related to its historical values. While in multivariate time series, the current value is not only related with its history but is also related to extra driving series. 

According to the type of models, time series analysis problems can be divided into two categories: linear models and nonlinear models. Traditional linear time series modelling includes auto regression (AR), moving average (MA), and auto regression moving average (ARMA) [8]. Nonlinear modelling includes auto regressive conditional heteroskedasticity (ARCH) [9], and general auto regressive conditional heteroskedasticity (GARCH) [10].

These traditional algorithms are easy to explain and have complete theoretical basis. They can give a prediction with a confidence interval and the model itself implies the correlation of the variables. Nevertheless, these traditional methods also have disadvantages such as weak predictive ability.

\section{Methods for Executing Spatio--Temporal Predictive Queries}
\label{Sec:RelatedWorksQueries}

A predictive spatio-temporal query, in particular a predictive range query has a query region $R$ and a time $t$, and asks about the predictions expected to be inside $R$ after time $t$ based on historical data (or previous knowledge).  In \cite{Akdere2011} the authors discusses the integration or extension of a RDBMS with a predictive component able to support data-driven predictive analytics. A predictive query and spatio-temporal predictive query, is defined as a traditional query using a declarative language that also has a predictive capability \cite{Hendawi2012}. 

They common uses for predictive queries: the support for predictive analytics to answer complex questions involving missing or future values, correlations, and trends, which  can be used to identify opportunities or threats
The predictive functionality can help build introspective services that assist in various data and resource management and optimization tasks (off-line or on-line predictive techniques). 
The  scope  of  this  production requires  more  efficient  querying development to  retrieve more accurate  information (better results)  within  the  shortest time  frame.

\section{Discussion}
Time-Series Classification is a difficult task and the traditional ML techniques have limitations due to the time dependency in the observations. Exists traditional approaches using sequential models like LSTM or RNN, but recent studies shows that the use of Deep Learning models are good options to obtain a decent performance and accuracy \cite{Fawaz2019}. 
