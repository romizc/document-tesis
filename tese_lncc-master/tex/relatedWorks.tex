\chapter{Related Works}\label{chapter_Related_Works}

As described on the previous chapter, this work attends primarily to the study and analysis of model selection techniques implemented in the production of models for predictive serving systems, our emphasis are phenomena considering univariate time-series analysis. 

Two main difficulties when modeling spatio--temporal data come from their size and from the complexity of the data generation process. \cite{}

We reviewed related studies for three main topics that are spatio--temporal modeling considering univariate time--series, model selection approaches considered in predictive serving system, and how to these techniques solve spatio--temporal predictive queries.

The classification, clustering, and searching through time series have important applications in many domains. In medicine EEG and ECG readings translate to time-series data collections with billions (even trillions) of points. In fact many research hospitals have trillions of points of EEG data. Other domains where large time series data collections are routine include gesture recognition \& user interface design, astronomy, robotics, geology, wildlife monitoring, security, and biometrics. 



This chapter concludes by discussing works on which constitute the most related research areas regarding this work.



\section{Time-Series Clustering and Classification}

% Time series clustering review 

During several decades the data mining community focused on time-series data (univariate and multivariate) clustering, which is a solution for classifying enormous data when there is not any early knowledge about classes, have developed techniques considering several different approaches to group them; partitioning-based, density-based, model-based, among others \cite{}. 

In accordance with the purpose of grouping the data, it is necessary to consider a measure of similarity between the time series. The traditional measure is the Euclidean distance, but for time series the focus is on the similarity of shape between them. Dynamic Time Warping (DTW) was originally used for speech re cognition and later introduced to other sequential data for similarity measurement. DTW has long been accepted as a good and efficient similarity measure for time series data \cite{}.






% Success Cases using DTW as a similarity measure






\section{Spatio--Temporal Modeling -- Uni-Variate Time Series Analysis}
\label{Sec:STModeling}
% Review the process of modelinng spatiotemporal data, focusing in simple methods that are very expensive 

% Modelos para problemas espacio-temporales:
In several domains, including climate science, neuroscience, social sciences, epidemiology, transportation, criminology, and Earth sciences, space and time are the main characteristic to consider in its observational data in the form of time--series. These aspects along with the technical advantages of nowadays, allow to domain specialists collect a vast amounts of data in order to analyze and study certain phenomenon with great detail \cite{}. 

The field of Spatio--Temporal modeling recently have 

\subsection{Model Selection}



\section{Methods for Predicting Spatio--Temporal Queries}
\label{Sec:RelatedWorksQueries}

A predictive spatio-temporal query, in particular a predictive range query has a query region $R$ and a time $t$, and asks about the predictions expected to be inside $R$ after time $t$ based on historical data (or previous knowledge).  In \cite{Akdere2011} the authors discusses the integration or extension of a RDBMS with a predictive component able to support data-driven predictive analytics. A predictive query and spatio-temporal predictive query, is defined as a traditional query using a declarative language that also has a predictive capability \cite{Hendawi2012}. 

They common uses for predictive queries: the support for predictive analytics to answer complex questions involving missing or future values, correlations, and trends, which  can be used to identify opportunities or threats 
The predictive functionality can help build introspective services that assist in various data and resource management and optimization tasks (off-line or on-line predictive techniques). 
The  scope  of  this  production requires  more  efficient  querying development to  retrieve more accurate  information (better results)  within  the  shortest time  frame.


%Related Works
%Spatio Temporal Modeling with emphasis in Univariate Time Series Analysis
%Literature review for the main challenges 
%Predictive Serving Systems (PSS)
%Definitions, components and model selection techniques during the serving models process.         
%Solving Spatio-Temporal Predictive Queries         
%Challenges and applicability    
%
%
\section{Discussion}
Time-Series Classification is a difficult task and the traditional ML techniques have limitations due to the time dependency in the observations. Exists traditional approaches using sequential models like LSTM or RNN, but recent studies shows that the use of Deep Learning models are good options to obtain a decent performance and accuracy \cite{Fawaz2019}. 
