\chapter{Related Works}
\label{chapter_Related_Works}

In the previous chapter we described the context and purpose of this work, the proposal of a methodology to process spatiotemporal predictive queries in a spatiotemporal domain with massive data (represented by univariate time series). This methodology considers the use of various techniques and algorithms for this type of data, we can consider three clear objectives: cluster, classify and predict the data trend. 

This chapter thus reviews related studies, by presenting current efforts of efficient techniques for time-series clustering and classification. Also, we review works for prediction methods in particular those focused on the temporal modeling for spatiotemporal data represented by univariate time-series. We conclude by discussing works on methods for model generation to process spatiotemporal predictive queries predictive models
% en ambientes donde existen modelos ya entrenados para diferentes regiones del dominio.

\section{Time-Series Clustering and Classification}
\label{Sec:ClusteringRelatedWorks}

% Time series clustering basics review 
During several decades the data mining community focused on time-series data, clustering is of particular interest in temporal data mining since it provides a mechanism to automatically find some structure in large data sets that would be otherwise difficult to summarize or visualize. In time were developed several clustering algorithms, criteria for evaluating the performance of results, and the measures to determine the similarity/dissimilarity between two time series being compared, either in the forms of raw data, extracted features, or some model parameters \cite{Liao2005, Aghabozorgi2015}.

We review works interested in methods concerning manipulating raw sequential data sets (univariate time series), using Partitioning Methods \cite{Kaufman2009}, in accordance with the purpose of grouping the data, it is necessary to consider a dis/similarity measure to compare the time series. Traditionally, the Euclidean distance were used to compare two time-series and measure the dis/similarity, but the nature of time--series ask for a more robust measure that is able to capture the temporal variability. Considering the dis/similarity focused on the shape between two time-series, the Dynamic Time Warping (DTW) measure, which originally was used for speech recognition, and later introduced to time-series for similarity measurement \cite{Sakoe1978}.

% Success Cases using DTW as a similarity measure
% The following works show the 

% Time Series Classification
Techniques and methods for Time--Series classification (TSC) in the last decades presented great advances due to the use of sequential models, in particular deep neural networks \cite{Fawaz2019}. In the same manner exists several approaches to process the time--series 
Geurts (2001) proposes to classify time series data based on combining local properties or patterns in the time series. Zhang et al. (2004) develop a representation method using wavelet decomposition that can automatically select the parameters for the classification task. They propose a nearest neighbor classification algorithm, using the derived appropriate scale. Kadous and Sammut (2005) use metafeature approach (i.e. recurring substructure) like local maxima in time series to generate classifiers. Similarly, Yang et al. (2005) focus on feature subset selection (FSS) based on common principal components, which is called CleVer, to retain the correlation information among original features. Classification is employed to evaluate the effectiveness of the selected subset of features. 
The use of neural networks for time-series classification

\section{Spatio--Temporal Modeling -- Uni-Variate Time Series Analysis}
\label{Sec:STModeling}
% Review the process of modelinng spatiotemporal data, focusing in simple methods that are very expensive 

% Modelos para problemas espacio-temporales:
In several domains, including climate science, neuroscience, social sciences, epidemiology, transportation, criminology, and Earth sciences, space and time are the main characteristic to consider in its observational data in the form of time--series. These aspects along with the technical advantages of nowadays, allow to domain specialists collect a vast amounts of data in order to analyze and study certain phenomenon with great detail \cite{}. 

Time series analysis has quite a long history. Techniques for statistical modeling and spectral analysis of real or complex-valued time series have been in use for more than fifty years \cite{Hyndman2006, Chatfield2019}. Weather forecasting, financial or stock market prediction and automatic process control have been some of the oldest and most studied applications of such time series analysis \cite{Box1976}. These applications saw the advent of an increased role for machine learning techniques like Hidden Markov Models and time-delay neural networks in time series analysis.


\subsection{Model Selection}
\label{Sec:STModelSelection}

In the context of spatio--temporal modeling 

\section{Methods for Executing Spatio--Temporal Predictive Queries}
\label{Sec:RelatedWorksQueries}

A predictive spatio-temporal query, in particular a predictive range query has a query region $R$ and a time $t$, and asks about the predictions expected to be inside $R$ after time $t$ based on historical data (or previous knowledge).  In \cite{Akdere2011} the authors discusses the integration or extension of a RDBMS with a predictive component able to support data-driven predictive analytics. A predictive query and spatio-temporal predictive query, is defined as a traditional query using a declarative language that also has a predictive capability \cite{Hendawi2012}. 

They common uses for predictive queries: the support for predictive analytics to answer complex questions involving missing or future values, correlations, and trends, which  can be used to identify opportunities or threats
The predictive functionality can help build introspective services that assist in various data and resource management and optimization tasks (off-line or on-line predictive techniques). 
The  scope  of  this  production requires  more  efficient  querying development to  retrieve more accurate  information (better results)  within  the  shortest time  frame.

\section{Discussion}
Time-Series Classification is a difficult task and the traditional ML techniques have limitations due to the time dependency in the observations. Exists traditional approaches using sequential models like LSTM or RNN, but recent studies shows that the use of Deep Learning models are good options to obtain a decent performance and accuracy \cite{Fawaz2019}. 
