\chapter{Using SPTA-TSA for Spatio-Temporal Temperature Dataset}
\label{apendiceA}

In Chapter \ref{chapter_Experimental_Results}, Section \ref{Sec:SPTA-TSA} we describe the diagrams of classes for the implementation of the SPTA-TSA. In this Apendix we describe the file structure and a user guide to use the SPTA-TSA for the Temperature dataset.

\section{Package File Tree}

As mentioned, the implementation of the package is entirely developed in Python 3 \footnote{Python 3: \url{http://www.python.org}}, the following is the file structure of the code.

\dirtree{%
	.1 spta-tsa.
	.2 docs.
	.3 source.
	.2 experiments.
	.3 arima.
	.3 auto\_arima.
	.3 baseline.
	.3 classifier.
	.3 clustering.
	.3 metadata.
	.3 region.
	.2 lstm.
%	.2 other.
	.2 outputs.
	.3 whole\_real\_brazil\_2014\_2014\_1spd.
	.4 dtw.
	.3 whole\_real\_brazil\_2014\_2014\_1spd\_scaled.
	.4 dtw.
    .5 regular\_k2.
	.6 auto-arima-start\_p1-start\_q1-max\_p3-max\_q3-dNone-stepwiseTrue.
	.5 regular\_k100.
	.2 pickle.
	.3 whole\_real\_brazil\_2014\_2014\_1spd\_scaled.
	.4 dtw.
	.5 kmedoids\_k100\_seed0\_lite.
	.2 raw.
	.2 scripts.
	.2 spta.
	.3 arima.
	.3 classifier.
	.3 clustering.
	.3 dataset.
	.3 dba.
	.3 distance.
	.3 kmedoids.
	.3 model.
	.3 region.
	.3 solver.
	.3 tests.
	.4 arima.
	.4 classifier.
	.4 clustering.
	.4 model.
	.4 region.
	.4 resources.
	.4 stub.
	.4 util.
	.3 util.
}					

\section{Dataset Extraction}

To extract the dataset we 



\section{Domain Partitioning}



\section{Generating Representative Predictive Models}


\section{Model Composition on Domain Partitioning}


\section{Generating Representative Predictive Model Classifier}


\section{Spatio--Temporal Predictive Query Execution}

