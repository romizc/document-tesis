\chapter{Using SPTA-TSA for Spatio-Temporal Temperature Dataset}
\label{apendiceA}

In this apendix we 

\section{Code and Files Structure}

%\dirtree{%
%	.1 spta-tsa.
%	.2 docs.
%	.3 source.
%	.2 experiments.
%	.3 arima.
%	.3 auto\_arima.
%	.3 baseline.
%	.3 classifier.
%	.3 clustering.
%	.3 metadata.
%	.3 region.
%	.2 lstm.
%	.2 other.
%	.2 outputs.
%	.3 whole\_real\_brazil\_2014\_2014\_1spd.
%	.4 dtw.
%	.3 whole\_real\_brazil\_2014\_2014\_1spd\_scaled.
%	.4 dtw.
%    .5 regular\_k2.
%	.6 auto-arima-start\_p1-start\_q1-max\_p3-max\_q3-dNone-stepwiseTrue.
%	.5 regular\_k100.
%	.2 pickle.
%	.3 whole\_real\_brazil\_2014\_2014\_1spd\_scaled.
%	.4 dtw.
%	.5 kmedoids\_k100\_seed0\_lite.
%	.2 raw.
%	.2 scripts.
%	.2 spta.
%	.3 arima.
%	.3 classifier.
%	.3 clustering.
%	.3 dataset.
%	.3 dba.
%	.3 distance.
%	.3 kmedoids.
%	.3 model.
%	.3 region.
%	.3 solver.
%	.3 tests.
%	.4 arima.
%	.4 classifier.
%	.4 clustering.
%	.4 model.
%	.4 region.
%	.4 resources.
%	.4 stub.
%	.4 util.
%	.3 util.
%}					

\section{Dataset Extraction}

To extract the dataset we 



\section{Domain Partitioning}



\section{Generating Representative Predictive Models}


\section{Model Composition on Domain Partitioning}


\section{Generating Representative Predictive Model Classifier}


\section{Spatio--Temporal Predictive Query Execution}

