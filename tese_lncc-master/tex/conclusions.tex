\chapter{Conclusions and Future Works}\label{chapter_Conclusions}

This final chapter contains the closing remarks of this work. Here, the most important contributions and findings are highlighted, with some additional discussion about the methodology, the analysis and the experiments. The chapter concludes with possible research lines that this work can open in the future.

\section{Results Summary}

The main objective of the experiments is to validate the proposed methodology, considering the case study of temperature prediction. According to the proposed experiments we can assert that the domain can be grouped according to some measure of similarity between the elements, and that these groups may be represented by a main element that generalizes the behavior of the group. 
%El objetivo principal de los experimentos realizados consiste en validar la metodologia propuesta, considerando el caso de estudio de la prediccion de temperatura. Seguindo los experimentos propuestos es posible afirmar que podemos agrupar el dominio de acuerdo a alguna medida de similaridad entre los elementos, y que estos grupos pueden ser representados por algun elemento principal que generaliza el comportamiento del grupo. 

Ese elemento generico a su vez, nos sirve para encontrar modelos predictivos. Los analisis realizados para asegurar la  que pueden ser usados para predicciones 
I

Considerando el caso de estudio prediccion de temperatura, fue posible validar la metodologia propuesta. Los experiementos 

\section{Main Contributions}



\section{Future Works}

Para el dominio de dados espacio-temporal considerado, la dimension temporal es la base de nuestra metodologia. Habiendo demostrado la utilidad del proceso completo, admitimos que existe espacio para mejorias considerables en el analisis presentado. 
\begin{itemize}
	\item Dado que hemos considerado la division del dominio mediante tecnicas de aprendizaje supervisionado, es posible ver que segun la naturaleza de los datos (autocorrelacionados y no estaticos) existe una dificultad en encontrar un numero ideal o un metodo para verificar si existe un tamano que presente un balance entre el costo de computar las particiones y la calidad de generalizacion de los elementos en cada grupo. 
	
	\item Los modelos predictivos considerados son modelos simples, que presentan , considerar modelos secuenciales 
	
	\item 
\end{itemize}