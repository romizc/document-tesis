\chapter{Conclusions and Future Works}\label{chapter_Conclusions}

This final chapter contains the closing remarks of this work. Here, the most important contributions and findings are highlighted, with some additional discussion about the methodology, the analysis and the experiments. The chapter concludes with possible research lines that this work can open in the future.

\section{Results Summary}

The main objective of the experiments is to evaluate the proposed methodology, considering the case study of temperature prediction. According to the proposed experiments we can assert that the domain can be grouped according to a shape-based (temporal) measure of similarity between elements, and these groups are represented by representative element that generalizes the behavior (temporal dimension) of the group. 

Considering these representative elements, the temporal predictive models built over them can also be considered as predictive models for the elements on each group. The performed analysis for the predictive quality of the representative temporal models, validate the 

With the study case, temperature forecast, we evaluate the validity of the proposed methodology. 

\section{Main Contributions}
\label{Sec:MainContributions}

We can consider the following contributions:
\begin{itemize}
    \item considerando modelos temporales sobre un dominio espacio-temporal es posible utilizar metodos con alta calidad predictiva, en este caso usamos modelos autoregresivos
    \item Desde el punto de vista computacional, considerar elementos que generalizan determinadas regiones del dominio representa una reduccion considerable en el costo de generacion de modelos. 

    \item 
\end{itemize}

A methodology for model selection in a spatio-temporal domain with high data volume.
In-depth experimental analysis to verify robustness of partitioning algorithm, forecast error analysis and validation using other baseline approaches.
Our multiclass model composition can produce forecast errors comparable to a naive approach, but with orders of magnitude more computationally efficient. 
Provide techniques to recognize and react to behaviors that can change over time due to external events and/or internal systematic changes in dynamics/distribution, making forecasting models obsolete.


\section{Future Works}

Para el dominio de dados espacio-temporal considerado, la dimension temporal es la base de nuestra metodologia. Habiendo demostrado la utilidad del proceso completo, admitimos que existe espacio para mejorias considerables en el analisis presentado. 
\begin{itemize}
	\item Dado que hemos considerado la division del dominio mediante tecnicas de aprendizaje supervisionado, es posible ver que segun la naturaleza de los datos (autocorrelacionados y no estaticos) existe una dificultad en encontrar un numero ideal o un metodo para verificar si existe un tamano que presente un balance entre el costo de computar las particiones y la calidad de generalizacion de los elementos en cada grupo. 
	
	\item Los modelos predictivos considerados son modelos simples, que presentan , considerar modelos secuenciales 
	
	\item 
\end{itemize}