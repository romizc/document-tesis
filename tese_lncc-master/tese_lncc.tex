%% tese_lncc.tex
%%
%% A última versão deste modelo está em
%%   https://github.com/equipe-customizacao-tese-lncc/tese_lncc
%%
%% Criado por:
%% Weslley da Silva Pereira
%% Lucas dos Santos Fernandez
%% Fortià Vila Verges
%%
%% Modificado por:
%% Equipe de customização - Fortià Vila Verges,
%%   Lucas dos Santos Fernandez, Weslley da Silva Pereira
%%
%% Este trabalho consiste de tese_lncc.tex,
%% abntex2lncc.sty e bibliografia.bib
%%

% ------------------------------------------------------------------------
% ------------------------------------------------------------------------
% Modelo de Trabalho Academico (tese de doutorado, dissertacao de
% mestrado e trabalhos monograficos em geral) em conformidade com
% ABNT NBR 14724:2011: Informacao e documentacao - Trabalhos academicos -
% Apresentacao
% ------------------------------------------------------------------------
% ------------------------------------------------------------------------

\PassOptionsToPackage{english,main=english}{babel}

\documentclass[
	% -- opções para output --
	%draft,				% simplificado, sem figuras
	final,				% final, mais demorado
	% -- opções da classe memoir --
	12pt,				% tamanho da fonte
	openright,			% capítulos começam em pág ímpar (insere página vazia caso preciso)
	oneside,			% para impressão em frente e verso use twoside
	a4paper,			% tamanho do papel.
	hyphens,            % Para quebrar linhas de url em footnotes
	% -- opções da classe abntex2 --
	%chapter=TITLE,		% títulos de capítulos convertidos em letras maiúsculas
	%section=TITLE,		% títulos de seções convertidos em letras maiúsculas
	%subsection=TITLE,	% títulos de subseções convertidos em letras maiúsculas
	%subsubsection=TITLE,% títulos de subsubseções convertidos em letras maiúsculas
	sumario=tradicional,% sumário tradicional, com tabulação
	% -- opções do pacote babel --
	english,			% idioma adicional para hifenização
	french,				% idioma adicional para hifenização
	spanish,			% idioma adicional para hifenização
	brazil				% o último idioma é o principal do documento (mude se precisar)
	]{abntex2}

% ---
% Pacotes básicos
% ---
\usepackage{lmodern}			% Usa a fonte Latin Modern			
\usepackage[T1]{fontenc}		% Selecao de codigos de fonte.
\usepackage[utf8]{inputenc}		% Codificacao do documento (conversão automática dos acentos)
\usepackage{lastpage}			% Usado pela Ficha catalográfica
\usepackage{indentfirst}		% Indenta o primeiro parágrafo de cada seção.
\usepackage{color}			    % Controle das cores
\usepackage{graphicx}			% Inclusão de gráficos
\usepackage{microtype} 			% para melhorias de justificação
\usepackage{amsthm,thmtools}	% Teoremas e outras definições matemáticas
\usepackage{abntex2lncc}		% Formatacao especifica do modelo do LNCC
\usepackage{amsmath}
\usepackage{amssymb}
\usepackage{enumitem}

% ---

% ---
% Pacotes de citações
% ---
\usepackage[brazilian,hyperpageref]{backref}	 % Paginas com as citações na bibl
\usepackage[alf]{abntex2cite}	% Citações padrão ABNT
% ---
		
% ---
% Pacotes adicionais, usados apenas no âmbito do Modelo Canônico do abnteX2
% ---
\usepackage{lipsum}				% para geração de dummy text
\usepackage{textcomp}
\usepackage{svg}
% ---

% ---
% CONFIGURAÇÕES DE PACOTES
% ---

% ---
% Configurações do pacote backref
% Usado sem a opção hyperpageref de backref
%\renewcommand{\backrefpagesname}{Citado na(s) página(s):~}
%% Texto padrão antes do número das páginas
%\renewcommand{\backref}{}
%% Define os textos da citação
%\renewcommand*{\backrefalt}[4]{
%	\ifcase #1 %
%		Nenhuma citação no texto.%
%	\or
%		Citado na página #2.%
%	\else
%		Citado #1 vezes nas páginas #2.%
%	\fi}%
%% ---

% % ---
% % Configurações do pacote backref no Ingles
% % Usado sem a opção hyperpageref de backref
 \renewcommand{\backrefpagesname}{Cited in:~}
 % Texto padrão antes do número das páginas
 \renewcommand{\backref}{}
 % Define os textos da citação
 \renewcommand*{\backrefalt}[4]{
 	\ifcase #1 %
 		Not cited.%
 	\or
 		Cited in page #2.%
 	\else
 		Cited #1 times in #2.%
 	\fi}%
 % ---

% ---
% Teoremas e outras definições matemáticas
\declaretheorem[style=definition,name=Definition, parent=chapter, qed=\textemdash]{definition}
\declaretheorem[style=definition,name=Observation, qed=\textnormal{\textemdash}]{remark}
\declaretheorem[style=plain,name=Theorem, qed=\textnormal{\textemdash}]{theorem}
\declaretheorem[style=plain,name=Proposition, qed=\textnormal{\textemdash}]{proposition}
\declaretheorem[style=plain,name=Corollary, qed=\textnormal{\textemdash}]{corollary}
\declaretheorem[style=plain,name=Axiom, qed=\textnormal{\textemdash}]{axiom}
\declaretheorem[style=plain,name=Lemma, qed=\textnormal{\textemdash}]{lemma}
% ---

% ---
% Comandos auxiliares para simplificar a inclusão
\newcommand{\texInput}[1]{\input{tex/#1}}
\newcommand{\texInclude}[1]{\include{tex/#1}}
\newcommand{\texIncludeonly}[1]{\includeonly{tex/#1}}
% ---

% ---
% Exemplo de uso do texIncludeonly que inclui somente a introdução no output
%\includeonly{introducao} % Usando o Makefile padrão
%\texIncludeonly{introducao} % Sem o Makefile padrão
% ---

% ---
% CONFIGURAÇÕES DE USUÁRIO
% ---
\DeclareMathOperator*{\argminA}{arg\,min} % Jan Hlavacek	

\usepackage{listings}
\usepackage{siunitx}
\usepackage{hyperref}
\usepackage{xcolor}
\usepackage{subfig}
%\usepackage{fontspec}
%\usepackage{minted}
\usepackage{multirow}
\usepackage{caption}
\usepackage{dirtree}


\newcommand{\Fab}[1]{\textbf{\textcolor{red}{[Fabio P.]: #1}}}
\newcommand{\Edu}[1]{\textbf{\textcolor{green}{[Eduardo O.]: #1}}}
\newcommand{\Rocio}[1]{\textbf{\textcolor{blue}{[Rocio Z.]: #1}}}


% ---
% Pasta principal de imagens e logo do LNCC
\graphicspath{{../Figures/}}
\logoLNCC{lncc}
%{\centering \includegraphics[scale=1.0]{lncc}}
% ---

% ---
% Tipo de trabalho (apenas uma das opções abaixo deve estar descomentada)
%\dissertacaoMestrado
\teseDoutorado
% ---

% ---
% Título
\titulo{An Spatial-Temporal Aware Model Selection for Time Series Analysis}
% Nome do aluno
\nomeAutor{Roc\'io Milagros}{Zorrilla Coz}
% Nome do orientador
\nomeOrientador{F\'abio Andr\'e}{Machado Porto}
% Coorientador(es)
\coorientador{Eduardo Ogasawara}
% \coorientador[Coorientadores:]{Coorientador 1 e Coorientador 2}
% ---

% ---
% Local
\local{Petrópolis, RJ - Brasil}
% Data
\data{Janeiro de 2021}
% Instituição
\instituicao{%
  Laboratório Nacional de Computação Científica
  \par
  Programa de Pós-Graduação em Modelagem Computacional}
% ---

% ---
% O preambulo deve conter o tipo do trabalho, o objetivo,
% o nome da instituição e a área de concentração
% portugues
\preambulo{\tipoTrabalho submetida ao corpo docente do Laboratório Nacional de Computação Científica como parte dos requisitos necessários para a obtenção do grau de \grau em Ciências em Modelagem Computacional.}
%ingles
%\preambulo{\tipoTrabalho submitted to the examining committee in partial fulfillment of the requirements for the degree of \grau of Sciences in Computational Modeling.}
% ---

% ---
% FICHA CATALOGRÁFICA
%
% Representa o código que sua tese/dissertação terá nos registros de  nossa biblioteca.
%
% Observação: Ao terminar de escrever sua tese/dissertação e a mesma for aprovada pela comissão de avaliação para a defesa, favor se dirigir a biblioteca.
% ---
\codebib{XXX.XXX}
\codetese{XXXX}
% ---

% ---
% Configurações de aparência do PDF final

\definecolor{blue}{RGB}{41,5,195}

% informações do PDF
\makeatletter
\hypersetup{
     	%pagebackref=true,
		pdftitle={\@title},
		pdfauthor={\@author},
    	pdfsubject={\imprimirpreambulo},
	    pdfcreator={LaTeX with abnTeX2},
		pdfkeywords={abnt}{latex}{abntex}{abntex2}{trabalho acadêmico},
		colorlinks=true,       		% false: boxed links; true: colored links
    	linkcolor=blue,          	% color of internal links
    	citecolor=blue,        		% color of links to bibliography
    	filecolor=magenta,      		% color of file links
		urlcolor=blue,
		bookmarksdepth=4
}
\makeatother
% ---

% ---
% Espaçamentos entre linhas e parágrafos
% ---

% O tamanho do parágrafo é dado por:
\setlength{\parindent}{1.3cm}

% Controle do espaçamento entre um parágrafo e outro:
\setlength{\parskip}{0.2cm}  % tente também \onelineskip

% ---
% compila o indice
% ---
\makeindex
% ---

% ----
% Início do documento
% ----

\begin{document}

% Seleciona o idioma do documento (conforme pacotes do babel)
\selectlanguage{english}
%\selectlanguage{brazil}

% Retira espaço extra obsoleto entre as frases.
\frenchspacing

% ----------------------------------------------------------
% ELEMENTOS PRÉ-TEXTUAIS
% ----------------------------------------------------------
% \pretextual

% ---
% Capa
% ---
\imprimircapa
% ---

% ---
% Folha de rosto
% (o * indica que haverá a ficha bibliográfica)
% ---
\imprimirfolhaderosto*
% ---

% ---
% Inserir a ficha bibliografica
% ---

% Isto é um exemplo de Ficha Catalográfica, ou ``Dados internacionais de
% catalogação-na-publicação''. Você pode utilizar este modelo como referência.
% Porém, provavelmente a biblioteca da sua universidade lhe fornecerá um PDF
% com a ficha catalográfica definitiva após a defesa do trabalho. Quando estiver
% com o documento, salve-o como PDF no diretório do seu projeto e substitua todo
% o conteúdo de implementação deste arquivo pelo comando abaixo:
%
% \begin{fichacatalografica}
%     \includepdf{fig_ficha_catalografica.pdf}
% \end{fichacatalografica}

\begin{fichacatalografica}
	\sffamily
	\vspace*{\fill}					% Posição vertical
	\begin{center}					% Minipage Centralizado
	\fbox{
	\begin{minipage}[c][5cm][t]{1.5cm}
	\small
	\imprimirCodeTese
	\end{minipage}
	\begin{minipage}[c][8cm][c]{13.5cm}		% Largura
	\small
	\imprimirUltimoSobrenome,{ }\imprimirNomeAutor
	%Sobrenome, Nome do autor
	
	\hspace{0.5cm} \imprimirtitulo{ }/ \imprimirautor. --
	\imprimirlocal, \imprimirdata-
	
	\hspace{0.5cm} \pageref{LastPage} p. : il. \pagColoridas ; 30 cm.\\
	
	\hspace{0.5cm} \imprimirOrientadoresRotulo~\imprimirorientador
	{ e }\imprimircoorientador\\
	
	\hspace{0.5cm}
	\parbox[t]{\textwidth}{\imprimirtipotrabalho~--~\imprimirinstituicao,
	\imprimirdata.}\\
	
	\hspace{0.5cm}
		1. Palavra-chave1.
		2. Palavra-chave2.
		2. Palavra-chave3.
		I. \imprimirUltimoSobrenomeOrientador,{ }\imprimirNomeOrientador.
		II. LNCC/MCTIC.
		III. \labelTitulo

	\begin{center}
		CDD: \imprimirCodeBib	
	\end{center}		
	\end{minipage}}
	
	\end{center}
\end{fichacatalografica}
% ---

% ---
% Inserir folha de aprovação
% ---

% Isto é um exemplo de Folha de aprovação, elemento obrigatório da NBR
% 14724/2011 (seção 4.2.1.3). Você pode utilizar este modelo até a aprovação
% do trabalho. Após isso, substitua todo o conteúdo deste arquivo por uma
% imagem da página assinada pela banca com o comando abaixo:
%
% \includepdf{folhadeaprovacao_final.pdf}
%
\begin{folhadeaprovacao}

  \begin{center}
    {\ABNTEXchapterfont\large\imprimirautor}

    \vspace*{\fill}\vspace*{\fill}
    \begin{center}
      \ABNTEXchapterfont\bfseries\Large\imprimirtitulo
    \end{center}
    \vspace*{\fill}

    \hspace{.45\textwidth}
    \begin{minipage}{.5\textwidth}
        \imprimirpreambulo
    \end{minipage}%
    \vspace*{\fill}
   \end{center}

   \aprovadaPor

   \assinatura{\textbf{Prof. \imprimirorientador,} \\ \presidenteDaBanca}
   \assinatura{\textbf{Prof. Agma, D. Sc.}}
   \assinatura{\textbf{Prof. Marta Lima de Queiroz Mattoso, D. Sc.}}
   \assinatura{\textbf{Prof. Jo\~ao Eduardo Ferreira, D. Sc.}}
   \assinatura{\textbf{Prof. Arthur Ziviani, Ph.D.}}

   \begin{center}
    \vspace*{0.5cm}
    {\large\imprimirlocal}
    \par
    {\large\imprimirdata}
    \vspace*{1cm}
  \end{center}

\end{folhadeaprovacao}
% ---

% ---
% Dedicatória
% ---
\begin{dedicatoria}
   \vspace*{\fill}
   \vspace*{10cm}
   \flushright
   \noindent
   \textbf{\dedicatorianame\\}
   \textit{To all....\\ to all fools.\\} \vspace*{\fill}
\end{dedicatoria}
% ---

% ---
% Agradecimentos
% ---
\begin{agradecimentos}
Gracias totales.

\end{agradecimentos}
% ---

% ---
% Epígrafe
% ---
\begin{epigrafe}
    \vspace*{\fill}
	\begin{flushright}
		\textit{``Título ou frase que serve de tema ao assunto ou para resumir\\ o sentido ou situar a motivação da obra.''\\
		(Referência para a epígrafe)}
	\end{flushright}
\end{epigrafe}
% ---

% ---
% RESUMOS
% ---

% resumo no idioma principal
\setlength{\absparsep}{18pt} % ajusta o espaçamento dos parágrafos do resumo
\begin{resumo}
 Segundo, o resumo deve ressaltar o
 objetivo, o método, os resultados e as conclusões do documento. A ordem e a extensão
 destes itens dependem do tipo de resumo (informativo ou indicativo) e do
 tratamento que cada item recebe no documento original. O resumo deve ser
 precedido da referência do documento, com exceção do resumo inserido no
 próprio documento. (\ldots) As palavras-chave devem figurar logo abaixo do
 resumo, antecedidas da expressão Palavras-chave:, separadas entre si por
 ponto e finalizadas também por ponto.

 \textbf{\palavrasChave}: latex. abntex. editoração de texto.
\end{resumo}

% resumo em inglês
\begin{resumo}[Abstract]
 \begin{otherlanguage*}{english}
 
 %Context
 A Spatio--Temporal Predictive Serving System enables users to express Predictive Queries that specify: a spatio--temporal region; a predictive variable and a evaluation metric. The outcome of a predictive query presents the values of the predictive variable on the specified region computed by predictive models that maximize the evaluation metric.    
 
 % Motivation
 In Spatio--Temporal domains, where datasets are represented by massive amounts of univariate time--series, traditional data processing and time--series analysis approaches to generate predictive models are costly in time and computational resources, although they present good predictive performance. 
 
 % Proposal
 In this work we propose a consistent methodology for spatio--temporal query processing, which aims to reduce the computational workload and time that would be consumed if we were to train a model on each element of a spatio–temporal domain, without losing performance of the models that attend the query. The focus of our methodology begins with the domain characterization to find groups represented by an element that generalize their temporal evolution, then using temporal auto-regressive models for these representative elements we analyze their predictive power to process spatio--temporal predictive queries.  
 
 %Results
 The computational experiments performed and a case study developed demonstrates the applicability of this methodology to generate predictive models in a spatio--temporal domain for query processing. We indicate that optimal predictive performance can be found when we evaluate a spatio-temporal query with a combination of shape-based domain discretization and temporal models.

   \textbf{Keywords}: Spatio--Temporal, Univariate Time--Series, Dynamic Time Warping (DTW), Auto--Regressive models.
 \end{otherlanguage*}
\end{resumo}

%% resumo em português-br
%\begin{resumo}[Resumo]
% \begin{otherlanguage*}{brazil}
%    Este é um resumo em português do Brasil.
%
%   \textbf{Palavras-chave}: latex. abntex. editoração de texto.
% \end{otherlanguage*}
%\end{resumo}

%% resumo em francês
%\begin{resumo}[Résumé]
% \begin{otherlanguage*}{french}
%    Il s'agit d'un résumé en français.
%
%   \textbf{Mots-clés}: latex. abntex. publication de textes.
% \end{otherlanguage*}
%\end{resumo}

%% resumo em espanhol
%\begin{resumo}[Resumen]
% \begin{otherlanguage*}{spanish}
%   Este es el resumen en español.
%
%   \textbf{Palabras clave}: latex. abntex. publicación de textos.
% \end{otherlanguage*}
%\end{resumo}
% ---

% ---
% inserir lista de figuras
% ---
\pdfbookmark[0]{\listfigurename}{lof}
\listoffigures*
\cleardoublepage
% ---

% ---
% inserir lista de tabelas
% ---
\pdfbookmark[0]{\listtablename}{lot}
\listoftables*
\cleardoublepage
% ---

% ---
% inserir lista de abreviaturas e siglas
% ---
\begin{siglas}
  \item[ABNT] Associação Brasileira de Normas Técnicas
  \item[abnTeX] ABsurdas Normas para TeX
\end{siglas}
% ---

% ---
% inserir lista de símbolos
% ---
\begin{simbolos}
  \item[$ \Gamma $] Letra grega Gama
  \item[$ \Lambda $] Lambda
  \item[$ \zeta $] Letra grega minúscula zeta
  \item[$ \in $] Pertence
\end{simbolos}
% ---

% ---
% inserir o sumario
% ---
\pdfbookmark[0]{\contentsname}{toc}
\tableofcontents*
\cleardoublepage
% ---



% ----------------------------------------------------------
% ELEMENTOS TEXTUAIS
% ----------------------------------------------------------
\textual

% ----------------------------------------------------------
% Capítulos
% ----------------------------------------------------------

% Inserir com o Makefile padrão
%\include{introducao}
%\include{capitulo2} 
%\include{capitulo3}
%\include{capitulo4}

% ---
% Alternativa sem o uso do Makefile padrão
\texInclude{introduction}
\texInclude{theoreticalFoundations}
\texInclude{relatedWorks} 
\texInclude{methodology}
\texInclude{experimentalResults}
%\texInclude{conclusions}
% ---

% ----------------------------------------------------------
% Finaliza a parte no bookmark do PDF
% para que se inicie o bookmark na raiz
% e adiciona espaço de parte no Sumário
% ----------------------------------------------------------
\phantompart

% ---
% Conclusão
% ---
%\include{conclusao} % Inserir com o Makefile padrão
\texInclude{conclusions} % Alternativa sem o uso do Makefile padrão
% ---

% ----------------------------------------------------------
% ELEMENTOS PÓS-TEXTUAIS
% ----------------------------------------------------------
\postextual
% ----------------------------------------------------------

% ----------------------------------------------------------
% Referências bibliográficas
% ----------------------------------------------------------
\bibliography{bibliografia}

% ----------------------------------------------------------
% Glossário
% ----------------------------------------------------------
%
% Consulte o manual da classe abntex2 para orientações sobre o glossário.
%
%\glossary

% ----------------------------------------------------------
% Apêndices
% ----------------------------------------------------------

% ---
% Inicia os apêndices
% ---
\begin{apendicesenv}

% Imprime uma página indicando o início dos apêndices
\partapendices

% Inclusão dos arquivos referentes aos apêndices
% ----------------------------------------------------------

% Inserir com o Makefile padrão
%\chapter{Using SPTA-TSA for Spatio-Temporal Temperature Dataset}
\label{apendiceA}

In this apendix we 

\section{Code and Files Structure}

%\dirtree{%
%	.1 spta-tsa.
%	.2 docs.
%	.3 source.
%	.2 experiments.
%	.3 arima.
%	.3 auto\_arima.
%	.3 baseline.
%	.3 classifier.
%	.3 clustering.
%	.3 metadata.
%	.3 region.
%	.2 lstm.
%	.2 other.
%	.2 outputs.
%	.3 whole\_real\_brazil\_2014\_2014\_1spd.
%	.4 dtw.
%	.3 whole\_real\_brazil\_2014\_2014\_1spd\_scaled.
%	.4 dtw.
%    .5 regular\_k2.
%	.6 auto-arima-start\_p1-start\_q1-max\_p3-max\_q3-dNone-stepwiseTrue.
%	.5 regular\_k100.
%	.2 pickle.
%	.3 whole\_real\_brazil\_2014\_2014\_1spd\_scaled.
%	.4 dtw.
%	.5 kmedoids\_k100\_seed0\_lite.
%	.2 raw.
%	.2 scripts.
%	.2 spta.
%	.3 arima.
%	.3 classifier.
%	.3 clustering.
%	.3 dataset.
%	.3 dba.
%	.3 distance.
%	.3 kmedoids.
%	.3 model.
%	.3 region.
%	.3 solver.
%	.3 tests.
%	.4 arima.
%	.4 classifier.
%	.4 clustering.
%	.4 model.
%	.4 region.
%	.4 resources.
%	.4 stub.
%	.4 util.
%	.3 util.
%}					

\section{Dataset Extraction}

To extract the dataset we 



\section{Domain Partitioning}



\section{Generating Representative Predictive Models}


\section{Model Composition on Domain Partitioning}


\section{Generating Representative Predictive Model Classifier}


\section{Spatio--Temporal Predictive Query Execution}


%\include{apendiceB}
%\include{apendiceC}

% ---
% Alternativa sem o uso do Makefile padrão
\texInclude{apendiceA}
% \texInclude{apendiceB}
% \texInclude{apendiceC}
% ---

% ----------------------------------------------------------

\end{apendicesenv}
% ---

% ----------------------------------------------------------
% Anexos
% ----------------------------------------------------------

% ---
% Inicia os anexos
% ---
\begin{anexosenv}

% Imprime uma página indicando o início dos anexos
%\partanexos

% Inclusão dos arquivos referentes aos anexos
% ----------------------------------------------------------

% Inserir com o Makefile padrão
%\chapter{SPTA-TSA Use}\label{anexoA}

\section{Dataset Extraction}




\section{Domain Characterization}

\section{Generating Predictors on Representatives}

\section{Solvers Execution}

\section{Query Processing}
%\include{anexoB}
%\include{anexoC}

% ---
% Alternativa sem o uso do Makefile padrão
%\texInclude{anexoA}
% \texInclude{anexoB}
% \texInclude{anexoC}
%% ---

% ----------------------------------------------------------

\end{anexosenv}

%---------------------------------------------------------------------
% INDICE REMISSIVO
%---------------------------------------------------------------------
\phantompart
\printindex
%---------------------------------------------------------------------

\end{document}
