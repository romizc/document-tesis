%% tese_lncc.tex
%%
%% A última versão deste modelo está em
%%   https://github.com/equipe-customizacao-tese-lncc/tese_lncc
%%
%% Criado por:
%% Weslley da Silva Pereira
%% Lucas dos Santos Fernandez
%% Fortià Vila Verges
%%
%% Modificado por:
%% Equipe de customização - Fortià Vila Verges,
%%   Lucas dos Santos Fernandez, Weslley da Silva Pereira
%%
%% Este trabalho consiste de tese_lncc.tex,
%% abntex2lncc.sty e bibliografia.bib
%%

% ------------------------------------------------------------------------
% ------------------------------------------------------------------------
% Modelo de Trabalho Academico (tese de doutorado, dissertacao de
% mestrado e trabalhos monograficos em geral) em conformidade com
% ABNT NBR 14724:2011: Informacao e documentacao - Trabalhos academicos -
% Apresentacao
% ------------------------------------------------------------------------
% ------------------------------------------------------------------------

\PassOptionsToPackage{english,main=english}{babel}

\documentclass[
	% -- opções para output --
	%draft,				% simplificado, sem figuras
	final,				% final, mais demorado
	% -- opções da classe memoir --
	12pt,				% tamanho da fonte
	openright,			% capítulos começam em pág ímpar (insere página vazia caso preciso)
	oneside,			% para impressão em frente e verso use twoside
	a4paper,			% tamanho do papel.
	hyphens,            % Para quebrar linhas de url em footnotes
	% -- opções da classe abntex2 --
	%chapter=TITLE,		% títulos de capítulos convertidos em letras maiúsculas
	%section=TITLE,		% títulos de seções convertidos em letras maiúsculas
	%subsection=TITLE,	% títulos de subseções convertidos em letras maiúsculas
	%subsubsection=TITLE,% títulos de subsubseções convertidos em letras maiúsculas
	sumario=tradicional,% sumário tradicional, com tabulação
	% -- opções do pacote babel --
	english,			% idioma adicional para hifenização
	french,				% idioma adicional para hifenização
	spanish,			% idioma adicional para hifenização
	brazil				% o último idioma é o principal do documento (mude se precisar)
	]{abntex2}

% ---
% Pacotes básicos
% ---
\usepackage{lmodern}			% Usa a fonte Latin Modern			
\usepackage[T1]{fontenc}		% Selecao de codigos de fonte.
\usepackage[utf8]{inputenc}		% Codificacao do documento (conversão automática dos acentos)
\usepackage{lastpage}			% Usado pela Ficha catalográfica
\usepackage{indentfirst}		% Indenta o primeiro parágrafo de cada seção.
\usepackage{color, colortbl}    % Controle das cores
\usepackage{graphicx}			% Inclusão de gráficos
\usepackage{microtype} 			% para melhorias de justificação
\usepackage{amsthm,thmtools}	% Teoremas e outras definições matemáticas
\usepackage{abntex2lncc}		% Formatacao especifica do modelo do LNCC
\usepackage{amsmath}
\usepackage{amssymb}
\usepackage{enumitem}

% ---

% ---
% Pacotes de citações
% ---
\usepackage[brazilian,hyperpageref]{backref}	 % Paginas com as citações na bibl
\usepackage[alf]{abntex2cite}	% Citações padrão ABNT
% ---
		
% ---
% Pacotes adicionais, usados apenas no âmbito do Modelo Canônico do abnteX2
% ---
\usepackage{lipsum}				% para geração de dummy text
\usepackage{textcomp}
\usepackage{svg}
% ---

% ---
% CONFIGURAÇÕES DE PACOTES
% ---

% ---
% Configurações do pacote backref
% Usado sem a opção hyperpageref de backref
%\renewcommand{\backrefpagesname}{Citado na(s) página(s):~}
%% Texto padrão antes do número das páginas
%\renewcommand{\backref}{}
%% Define os textos da citação
%\renewcommand*{\backrefalt}[4]{
%	\ifcase #1 %
%		Nenhuma citação no texto.%
%	\or
%		Citado na página #2.%
%	\else
%		Citado #1 vezes nas páginas #2.%
%	\fi}%
%% ---

% % ---
% % Configurações do pacote backref no Ingles
% % Usado sem a opção hyperpageref de backref
 \renewcommand{\backrefpagesname}{Cited in:~}
 % Texto padrão antes do número das páginas
 \renewcommand{\backref}{}
 % Define os textos da citação
 \renewcommand*{\backrefalt}[4]{
 	\ifcase #1 %
 		Not cited.%
 	\or
 		Cited in page #2.%
 	\else
 		Cited #1 times in #2.%
 	\fi}%
 % ---

% ---
% Teoremas e outras definições matemáticas
\declaretheorem[style=definition,name=Definition, parent=chapter, qed=\textemdash]{definition}
\declaretheorem[style=definition,name=Observation, qed=\textnormal{\textemdash}]{remark}
\declaretheorem[style=plain,name=Theorem, qed=\textnormal{\textemdash}]{theorem}
\declaretheorem[style=plain,name=Proposition, qed=\textnormal{\textemdash}]{proposition}
\declaretheorem[style=plain,name=Corollary, qed=\textnormal{\textemdash}]{corollary}
\declaretheorem[style=plain,name=Axiom, qed=\textnormal{\textemdash}]{axiom}
\declaretheorem[style=plain,name=Lemma, qed=\textnormal{\textemdash}]{lemma}
% ---

% ---
% Comandos auxiliares para simplificar a inclusão
\newcommand{\texInput}[1]{\input{tex/#1}}
\newcommand{\texInclude}[1]{\include{tex/#1}}
\newcommand{\texIncludeonly}[1]{\includeonly{tex/#1}}
% ---

% ---
% Exemplo de uso do texIncludeonly que inclui somente a introdução no output
%\includeonly{introducao} % Usando o Makefile padrão
%\texIncludeonly{introducao} % Sem o Makefile padrão
% ---

% ---
% CONFIGURAÇÕES DE USUÁRIO
% ---
\DeclareMathOperator*{\argminA}{arg\,min} % Jan Hlavacek	

\usepackage{listings}
\usepackage{siunitx}
\usepackage{hyperref}
\usepackage{subfig}
\usepackage{algorithm}
\usepackage{algpseudocode}
\usepackage{multirow}
\usepackage{caption}
\usepackage{dirtree}
\usepackage{mismath}
\usepackage{tikz}
\tikzset{hl/.style={
    set fill color=red!10,
    set border color=red!60,
  },
}

\definecolor{MyRed}{RGB}{248,69,69}

\newcommand{\Fab}[1]{\textbf{\textcolor{red}{[Fabio P.]: #1}}}
\newcommand{\Edu}[1]{\textbf{\textcolor{green}{[Eduardo O.]: #1}}}
\newcommand{\Rocio}[1]{\textbf{\textcolor{blue}{[Rocio Z.]: #1}}}


% ---
% Pasta principal de imagens e logo do LNCC
\graphicspath{{../Figures/}}
\logoLNCC{lncc}
%{\centering \includegraphics[scale=1.0]{lncc}}
% ---

% ---
% Tipo de trabalho (apenas uma das opções abaixo deve estar descomentada)
%\dissertacaoMestrado
\teseDoutorado
% ---

% ---
% Título
\titulo{A Spatial-Temporal Aware Model Selection for Time Series Analysis}
% Nome do aluno
\nomeAutor{Roc\'io Milagros}{Zorrilla Coz}
% Nome do orientador
\nomeOrientador{F\'abio Andr\'e}{Machado Porto}
% Coorientador(es)
\coorientador{Eduardo Ogasawara}
% \coorientador[Coorientadores:]{Coorientador 1 e Coorientador 2}
% ---

% ---
% Local
\local{Petrópolis, RJ - Brasil}
% Data
\data{May, 2021}
% Instituição
\instituicao{%
  Laboratório Nacional de Computação Científica
  \par
  Programa de Pós-Graduação em Modelagem Computacional}
% ---

% ---
% O preambulo deve conter o tipo do trabalho, o objetivo,
% o nome da instituição e a área de concentração
% portugues
%\preambulo{\tipoTrabalho submetida ao corpo docente do Laboratório Nacional de Computação Científica como parte dos requisitos necessários para a obtenção do grau de \grau em Ciências em Modelagem Computacional.}
%ingles
\preambulo{\tipoTrabalho submitted to the examining committee in partial fulfillment of the requirements for the degree of \grau of Sciences in Computational Modeling.}
% ---

% ---
% FICHA CATALOGRÁFICA
%
% Representa o código que sua tese/dissertação terá nos registros de  nossa biblioteca.
%
% Observação: Ao terminar de escrever sua tese/dissertação e a mesma for aprovada pela comissão de avaliação para a defesa, favor se dirigir a biblioteca.
% ---
\codebib{XXX.XXX}
\codetese{XXXX}
% ---

% ---
% Configurações de aparência do PDF final

\definecolor{blue}{RGB}{41,5,195}

% informações do PDF
\makeatletter
\hypersetup{
     	%pagebackref=true,
		pdftitle={\@title},
		pdfauthor={\@author},
    	pdfsubject={\imprimirpreambulo},
	    pdfcreator={LaTeX with abnTeX2},
		pdfkeywords={abnt}{latex}{abntex}{abntex2}{trabalho acadêmico},
		colorlinks=true,       		% false: boxed links; true: colored links
    	linkcolor=blue,          	% color of internal links
    	citecolor=blue,        		% color of links to bibliography
    	filecolor=magenta,      		% color of file links
		urlcolor=blue,
		bookmarksdepth=4
}
\makeatother
% ---

% ---
% Espaçamentos entre linhas e parágrafos
% ---

% O tamanho do parágrafo é dado por:
\setlength{\parindent}{1.3cm}

% Controle do espaçamento entre um parágrafo e outro:
\setlength{\parskip}{0.2cm}  % tente também \onelineskip

% ---
% compila o indice
% ---
\makeindex
% ---

% ----
% Início do documento
% ----

\begin{document}

% Seleciona o idioma do documento (conforme pacotes do babel)
\selectlanguage{english}
%\selectlanguage{brazil}

% Retira espaço extra obsoleto entre as frases.
\frenchspacing

% ----------------------------------------------------------
% ELEMENTOS PRÉ-TEXTUAIS
% ----------------------------------------------------------
% \pretextual

% ---
% Capa
% ---
\imprimircapa
% ---

% ---
% Folha de rosto
% (o * indica que haverá a ficha bibliográfica)
% ---
\imprimirfolhaderosto*
% ---

% ---
% Inserir a ficha bibliografica
% ---

% Isto é um exemplo de Ficha Catalográfica, ou ``Dados internacionais de
% catalogação-na-publicação''. Você pode utilizar este modelo como referência.
% Porém, provavelmente a biblioteca da sua universidade lhe fornecerá um PDF
% com a ficha catalográfica definitiva após a defesa do trabalho. Quando estiver
% com o documento, salve-o como PDF no diretório do seu projeto e substitua todo
% o conteúdo de implementação deste arquivo pelo comando abaixo:
%
% \begin{fichacatalografica}
%     \includepdf{fig_ficha_catalografica.pdf}
% \end{fichacatalografica}

\begin{fichacatalografica}
	\sffamily
	\vspace*{\fill}					% Posição vertical
	\begin{center}					% Minipage Centralizado
	\fbox{
	\begin{minipage}[c][5cm][t]{1.5cm}
	\small
	\imprimirCodeTese
	\end{minipage}
	\begin{minipage}[c][8cm][c]{13.5cm}		% Largura
	\small
	\imprimirUltimoSobrenome,{ }\imprimirNomeAutor
	%Sobrenome, Nome do autor
	
	\hspace{0.5cm} \imprimirtitulo{ }/ \imprimirautor. --
	\imprimirlocal, \imprimirdata-
	
	\hspace{0.5cm} \pageref{LastPage} p. : il. \pagColoridas ; 30 cm.\\
	
	\hspace{0.5cm} \imprimirOrientadoresRotulo~\imprimirorientador
	{ e }\imprimircoorientador\\
	
	\hspace{0.5cm}
	\parbox[t]{\textwidth}{\imprimirtipotrabalho~--~\imprimirinstituicao,
	\imprimirdata.}\\
	
	\hspace{0.5cm}
		1. Palavra-chave1.
		2. Palavra-chave2.
		2. Palavra-chave3.
		I. \imprimirUltimoSobrenomeOrientador,{ }\imprimirNomeOrientador.
		II. LNCC/MCTIC.
		III. \labelTitulo

	\begin{center}
		CDD: \imprimirCodeBib	
	\end{center}		
	\end{minipage}}
	
	\end{center}
\end{fichacatalografica}
% ---

% ---
% Inserir folha de aprovação
% ---

% Isto é um exemplo de Folha de aprovação, elemento obrigatório da NBR
% 14724/2011 (seção 4.2.1.3). Você pode utilizar este modelo até a aprovação
% do trabalho. Após isso, substitua todo o conteúdo deste arquivo por uma
% imagem da página assinada pela banca com o comando abaixo:
%
% \includepdf{folhadeaprovacao_final.pdf}
%
\begin{folhadeaprovacao}

  \begin{center}
    {\ABNTEXchapterfont\large\imprimirautor}

    \vspace*{\fill}\vspace*{\fill}
    \begin{center}
      \ABNTEXchapterfont\bfseries\Large\imprimirtitulo
    \end{center}
    \vspace*{\fill}

    \hspace{.45\textwidth}
    \begin{minipage}{.5\textwidth}
        \imprimirpreambulo
    \end{minipage}%
    \vspace*{\fill}
   \end{center}

   \aprovadaPor

   \assinatura{\textbf{Prof. \imprimirorientador,} \\ \presidenteDaBanca}
   \assinatura{\textbf{Prof. Agma Juci Machado Traina, D. Sc.}}
   \assinatura{\textbf{Prof. Fl\'avia Coimbra Delicato, D. Sc.}}
   \assinatura{\textbf{Prof. Jo\~ao Eduardo Ferreira, D. Sc.}}
   \assinatura{\textbf{Prof. Ant\^onio Tadeu Azevedo Gomes, D. Sc.}}
 
 \assinatura{\textbf{Prof. Bruno Richard Schulze, D. Sc.}}

   \begin{center}
    \vspace*{0.5cm}
    {\large\imprimirlocal}
    \par
    {\large\imprimirdata}
    \vspace*{1cm}
  \end{center}

\end{folhadeaprovacao}
% ---

% ---
% Dedicatória
% ---
\begin{dedicatoria}
   \vspace*{\fill}
   \vspace*{10cm}
   \flushright
   \noindent
   \textbf{\dedicatorianame\\}
   \textit{To Giacomo and Clarissa, their love and support keep me going.\\} \vspace*{\fill}
\end{dedicatoria}
% ---

% ---
% Agradecimentos
% ---
\begin{agradecimentos}
I would like to thank everyone who helped me to conduct this research. Mostly to my supervisors for all the valuable advice and knowledge they shared with me. I am very thankful for the guidance of Prof. Fabio Porto and Prof. Eduardo Ogasawara, under which I have learned a lot about research.

I also need to thank my parents, brother and sisters for their support. Special thanks to Giacomo and Clarissa whom have always pushed me and motivated me throughout this time.

Last but not least, I would like to express my gratitude to the National Laboratory of Scientific Computing (LNCC), and Data Extreme Laboratory (DEXL) for the opportunity. 

\end{agradecimentos}
% ---

% ---
% Epígrafe
% ---
\begin{epigrafe}
    \vspace*{\fill}
	\begin{flushright}
		\textit{``Try, Try Again''\\
		`Tis a lesson you should heed, \\
		Try, try again;\\
		If at first you don't succeed,\\
		Try, try again;\\
		Then your courage should appear, \\
		For, if you will persevere, \\
		You will conquer, never fear;\\
		Try, try again.\\
		(William J. Bennett, The Children's Book of Virtues.)}
	\end{flushright}
\end{epigrafe}
% ---

% ---
% RESUMOS
% ---

% resumo no idioma principal
\setlength{\absparsep}{18pt} % ajusta o espaçamento dos parágrafos do resumo
\begin{resumo}[Resumo]

Um Sistema de Serviço para Predição Espaço-Temporal é uma solução baseada na implantação de modelos pré-treinados, que permite aos usuários expressar Consultas Preditivas. As consultas preditivas espaço-temporais abrangem uma região espaço-temporal, uma variável preditiva e uma métrica de avaliação. O resultado dessa consulta contém os valores da variável preditiva na região especificada, calculada por modelos preditivos, procurando ao mesmo tempo maximizar a métrica de avaliação.
 
 % De motivação
 Em domínios espaço-temporais, nos quais conjuntos de dados são representados por quantidades massivas de séries temporais, as abordagens tradicionais de processamento de dados e da análise de séries temporais tendem a gerar modelos preditivos que visam a acurácia na predição, resultando em tempos de execução e utilização de recursos computacionais elevados.

 % Proposta
 Neste trabalho, é proposta uma metodologia passo a passo para a avaliação de consultas preditivas espaço-temporais, que visa reduzir a carga de trabalho computacional e o tempo que seria consumido se fosse treinado um modelo em cada um dos elementos de um domínio espaço--temporal. Isso é obtido escolhendo cuidadosamente os modelos preditivos a serem utilizados para inferência em cada elemento, dada uma consulta preditiva espaço--temporal.
 
 A metodologia proposta está composta de três etapas offline e uma etapa online: (1) o particionamento do domínio, baseado em técnicas de agrupamento com elementos representativos; (2) a geração de modelos preditivos temporais para os representantes; (3) um processo de classificação de série temporal que alavanca relacionamentos subjacentes entre modelos representativos e o particionamento do domínio; (4) um processo de inferência online que utiliza o classificador de séries temporais para compor modelos e calcular uma consulta preditiva espaço-temporal.
 
 %Resultados
 A fim de avaliar a aplicabilidade da metodologia proposta, é utilizado um estudo de caso relacionado à previsão de temperatura, usando dados históricos e modelos autorregressivos. Através dos resultados de experimentos computacionais, é mostrado que é possível alcançar uma qualidade preditiva comparável usando uma composição de modelos baseada em representantes de grupos, com apenas uma fração do custo computacional.

 \textbf{\palavrasChave}: Espaço-temporal, séries temporais, agrupamento de séries temporais, classificação de séries temporais, modelos autorregressivos.
\end{resumo}

% resumo em inglês
\begin{resumo}[Abstract]
 \begin{otherlanguage*}{english}
 %Context
A Spatio-Temporal Predictive Serving System is a solution based on pre-trained models that enables users to express Predictive Queries. Spatio-temporal Predictive Queries encompass a spatio-temporal region, a predictive variable, and an evaluation metric. The outcome of such query presents the values of the predictive variable on the specified region computed by predictive models while striving to maximize the evaluation metric.
 
 % Motivation
 In Spatio-Temporal domains, where datasets are represented by massive amounts of univariate time-series, traditional data processing, and time-series analysis approaches tend to generate predictive models that aim for predictive accuracy, at the cost of large running times and high utilization of computational resources.

 % Proposal
 In this work, we propose a step-by-step methodology for evaluating spatio-temporal predictive queries that aims to reduce the computational workload and time consumed if we were to train a model on each element of a spatio–temporal domain.  It is achieved by carefully choosing the predictive models for inference at each element, given a spatio-temporal predictive query.
 
 Our methodology has three offline steps and an online step: (1) the domain partitioning, based on clustering techniques with representative elements; (2) the generation of temporal predictive models for the representatives; (3) a time series classification process that leverages underlying relationships between representative models and domain partitioning; (4) an online inference process that uses the time series classifier to compose models and compute a spatio-temporal predictive query.
 
 %Results
 In order to evaluate the applicability of the proposed methodology, we use a case study for temperature forecasting using historical data and auto-regressive models. Results from computational experiments show that it is possible to achieve comparable predictive quality using a model composition based on cluster representatives, with a fraction of the computational cost. 
 
   \textbf{Keywords}: Spatio-Temporal, Time--Series, Time--Series Clustering, Time--Series Classification, Auto--Regressive models.
 \end{otherlanguage*}
\end{resumo}

%% resumo em português-br
%\begin{resumo}[Resumo]
% \begin{otherlanguage*}{brazil}
%    Este é um resumo em português do Brasil.
%
%   \textbf{Palavras-chave}: latex. abntex. editoração de texto.
% \end{otherlanguage*}
%\end{resumo}

%% resumo em francês
%\begin{resumo}[Résumé]
% \begin{otherlanguage*}{french}
%    Il s'agit d'un résumé en français.
%
%   \textbf{Mots-clés}: latex. abntex. publication de textes.
% \end{otherlanguage*}
%\end{resumo}

%% resumo em espanhol
%\begin{resumo}[Resumen]
% \begin{otherlanguage*}{spanish}
%   Este es el resumen en español.
%
%   \textbf{Palabras clave}: latex. abntex. publicación de textos.
% \end{otherlanguage*}
%\end{resumo}
% ---

% ---
% inserir lista de figuras
% ---
\pdfbookmark[0]{\listfigurename}{lof}
\listoffigures*
\cleardoublepage
% ---

% ---
% inserir lista de tabelas
% ---
\pdfbookmark[0]{\listtablename}{lot}
\listoftables*
\cleardoublepage
% ---

% ---
% inserir lista de abreviaturas e siglas
% ---
\begin{siglas}
  \item[AIC] Akaike Information Criterion
  \item[ARIMA] Auto Regressive Integrated Moving Average
  \item[CFSR] Climate Forecast System Reanalysis
  \item[CNN] Convolutional Neural Network
  \item[DNN] Deep Neural Network
  \item[DTW] Dynamic Time Warping
  \item[GIS] Geographic Information Science  
  \item[kNN] $k$ Nearest Neighbor
  \item[LSTM] Long--Short Term Memory
  \item[MAPE] Mean Absolute Percentage Error
  \item[ML] Machine Learning
  \item[MSE] Mean Squared Error
  \item[PAM] Partition Around Medoids
  \item[RDBMS] Relational Database Management System
  \item[ReLU] Rectified Linear Unit
  \item[RMSE] Root Mean Squared Error
  \item[RNN] Recurrent Neural Network
  \item[sMAPE] Symmetric Mean Absolute Percentage Error
  \item[SPT-TSA] Spatio-Temporal Tool for Time Series Analysis
  \item[SSE] Sum of Squares Error
  \item[ST] Spatio--temporal
  \item[TSCD] Time Series Classification Dataset
  \item[WSS] Within--cluster Sum of Squares
\end{siglas}
% ---

% ---
% inserir lista de símbolos
% ---
% \begin{simbolos}
%   \item[$ \Gamma $] Letra grega Gama
%   \item[$ \Lambda $] Lambda
%   \item[$ \zeta $] Letra grega minúscula zeta
%   \item[$ \in $] Pertence
% \end{simbolos}
% ---

% ---
% inserir o sumario
% ---
\pdfbookmark[0]{\contentsname}{toc}
\tableofcontents*
\cleardoublepage
% ---



% ----------------------------------------------------------
% ELEMENTOS TEXTUAIS
% ----------------------------------------------------------
\textual

% ----------------------------------------------------------
% Capítulos
% ----------------------------------------------------------

% Inserir com o Makefile padrão
%\include{introducao}
%\include{capitulo2} 
%\include{capitulo3}
%\include{capitulo4}

% ---
% Alternativa sem o uso do Makefile padrão
\texInclude{1_introduction}
\texInclude{2_theoreticalFoundations}
\texInclude{3_relatedWorks} 
\texInclude{4_methodology}
% \texInclude{5_implementation}
\texInclude{6_experimentalResults}
%\texInclude{conclusions}
% ---

% ----------------------------------------------------------
% Finaliza a parte no bookmark do PDF
% para que se inicie o bookmark na raiz
% e adiciona espaço de parte no Sumário
% ----------------------------------------------------------
\phantompart

% ---
% Conclusão
% ---
%\include{conclusao} % Inserir com o Makefile padrão
\texInclude{7_conclusions} % Alternativa sem o uso do Makefile padrão
% ---

% ----------------------------------------------------------
% ELEMENTOS PÓS-TEXTUAIS
% ----------------------------------------------------------
\postextual
% ----------------------------------------------------------

% ----------------------------------------------------------
% Referências bibliográficas
% ----------------------------------------------------------
\bibliography{bibliografia}

% ----------------------------------------------------------
% Glossário
% ----------------------------------------------------------
%
% Consulte o manual da classe abntex2 para orientações sobre o glossário.
%
%\glossary

% ----------------------------------------------------------
% Apêndices
% ----------------------------------------------------------

% ---
% Inicia os apêndices
% ---
\begin{apendicesenv}

% Imprime uma página indicando o início dos apêndices
\partapendices

% Inclusão dos arquivos referentes aos apêndices
% ----------------------------------------------------------

% Inserir com o Makefile padrão
% \include{5_implementation}
%\include{apendiceB}
%\include{apendiceC}

% ---
% Alternativa sem o uso do Makefile padrão
\texInclude{5_implementation}
% \texInclude{apendiceB}
% \texInclude{apendiceC}
% ---

% ----------------------------------------------------------

\end{apendicesenv}
% ---

% ----------------------------------------------------------
% Anexos
% ----------------------------------------------------------

% ---
% Inicia os anexos
% ---
% \begin{anexosenv}

% Imprime uma página indicando o início dos anexos
%\partanexos

% Inclusão dos arquivos referentes aos anexos
% ----------------------------------------------------------

% Inserir com o Makefile padrão
%\chapter{SPTA-TSA Use}\label{anexoA}

\section{Dataset Extraction}




\section{Domain Characterization}

\section{Generating Predictors on Representatives}

\section{Solvers Execution}

\section{Query Processing}
%\include{anexoB}
%\include{anexoC}

% ---
% Alternativa sem o uso do Makefile padrão
%\texInclude{anexoA}
% \texInclude{anexoB}
% \texInclude{anexoC}
%% ---

% ----------------------------------------------------------

% \end{anexosenv}

%---------------------------------------------------------------------
% INDICE REMISSIVO
%---------------------------------------------------------------------
\phantompart
\printindex
%---------------------------------------------------------------------

\end{document}
